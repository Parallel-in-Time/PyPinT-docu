\documentclass[%
  english,
  hyperref={pdfpagelabels=false},
  aspectratio=1610]{beamer}

\usepackage{lmodern}  % nicer fonts
\usepackage[utf8]{inputenc}
\usepackage[T1]{fontenc}

\mode<handout>{
  \usepackage{pgf}
  \usepackage{pgfpages}
  \pgfpagesuselayout{2 on 1}[a4paper,border shrink=5mm]
}
\mode<presentation>{
  \usetheme{Juelich}
}

\usepackage[english]{babel}
\usepackage[scaled]{helvet}
\usepackage{graphics}

\usepackage{listings}  % for source code display

\usepackage{hyperref}

\title{PyPinT}
\subtitle{Towards a framework for rapid prototyping of iterative parallel-in-time algorithms}
\author{Dieter Moser, Torbjörn Klatt, Dr. Robert Speck
  <\href{mailto:d.moser@fz-juelich.de}{\{d.moser,t.klatt,r.speck\}@fz-juelich.de}>%
}
\institute{3rd Workshop on Parallel-in-Time Integration Methods}
\date{May 28, 2014}
\setbeamertemplate{slide counter}[showall]


\begin{document}
\maketitle

\begin{frame}
  \frametitle{Overview}
  \begin{enumerate}
    \item Recap of existing Parallel-in-Time Algorithms
    \item The \emph{PyPinT} Framework Explained
    \item Goals
    \item Proof of Concept --- Examples Analyzed
  \end{enumerate}
\end{frame}


%%%%%%%%%%%%%%%%%%%%%%%%%%%%%%%%%%%%%%%%%%%%%%%%%%%%%%%%%%%%%%%%%%%%%%%%%%%%%%%%
\part{Existing Parallel-in-Time Approaches}
\makepart

\begin{frame}
  \frametitle{Parareal}
\end{frame}

\begin{frame}
  \frametitle{RIDC}
\end{frame}

\begin{frame}
  \frametitle{PFASST}
\end{frame}


%%%%%%%%%%%%%%%%%%%%%%%%%%%%%%%%%%%%%%%%%%%%%%%%%%%%%%%%%%%%%%%%%%%%%%%%%%%%%%%%
\part{The PyPinT Framework Explained}
\makepart

\begin{frame}
  \frametitle{Basic Concept}
  \begin{itemize}
    \item \emph{Python} as language of choice\\
      {\scriptsize for ease of use and extensibility (cf. \emph{NumPy}, \emph{SciPy})}
    \item well-conceived and intuitive abstract interfaces\\
      {\scriptsize for reusable code ensuring DRY principle}
    \item modular building blocks\\
      {\scriptsize for fast exchange of algorithms' building blocks}
    \item integrated analyzation tools\\
      {\scriptsize for introspection and plotting (cf. \emph{matplotlib})}
    \item usage of a sophisticated testing framework\\
      {\scriptsize nobody writes bug-free code}
  \end{itemize}
  
  \begin{picture}(1,1)
    \put(260,150){\includegraphics[height=1.25cm]{src/python_logo.png}}
    \put(290,135){\includegraphics[height=0.75cm]{src/numpy_logo.png}}
    \put(325,135){\includegraphics[height=0.75cm]{src/scipy_logo.png}}
    \put(290,80){\includegraphics[height=1.1cm]{src/puzzle.png}}
    \put(255,50){\includegraphics[height=0.75cm]{src/matplotlib_logo.png}}
  \end{picture}
\end{frame}

\begin{frame}
  \frametitle{Modules}
\end{frame}


%%%%%%%%%%%%%%%%%%%%%%%%%%%%%%%%%%%%%%%%%%%%%%%%%%%%%%%%%%%%%%%%%%%%%%%%%%%%%%%%
\part{Goals}
\makepart


%%%%%%%%%%%%%%%%%%%%%%%%%%%%%%%%%%%%%%%%%%%%%%%%%%%%%%%%%%%%%%%%%%%%%%%%%%%%%%%%
\part{Proof of Concept}
\makepart


%%%%%%%%%%%%%%%%%%%%%%%%%%%%%%%%%%%%%%%%%%%%%%%%%%%%%%%%%%%%%%%%%%%%%%%%%%%%%%%%
\begin{frame}
  \frametitle{~}
  \begin{center}
    {\huge Thank you for your attention!}\par
    \bigskip
    \bigskip
    \bigskip
    {\Large Questions?}\par
    {\scriptsize\color{fzjgray50}(now or later)}\par
    \bigskip
    {\scriptsize \emph{PyPinT} is on \includegraphics[height=1em]{GitHub-Mark-32px.png}~GitHub: \url{https://github.com/Parallel-in-Time/PyPinT}}
    \bigskip
    \begin{columns}
      \tiny
      \begin{column}{0.09\textwidth}
      \end{column}
      \begin{column}{0.3\textwidth}
        Dieter Moser\\
        Juelich Supercomputing Centre\\
        Building 16.3, Room 022\\
        Tel: +49~2461~61~96453\\
        eMail: \href{mailto:d.moser@fz-juelich.de}{d.moser@fz-juelich.de}
      \end{column}
      \begin{column}{0.3\textwidth}
        Torbjörn Klatt\\
        Juelich Supercomputing Centre\\
        Building 16.3, Room 022\\
        Tel: +49~2461~61~96452\\
        eMail: \href{mailto:t.klatt@fz-juelich.de}{t.klatt@fz-juelich.de}
      \end{column}
      \begin{column}{0.3\textwidth}
        Dr. Robert Speck~*\\
        Juelich Supercomputing Centre\\
        Building 16.3, Room 131\\
        Tel: +49~2461~61~1644\\
        eMail: \href{mailto:r.speck@fz-juelich.de}{r.speck@fz-juelich.de}
      \end{column}
      \begin{column}{0.01\textwidth}
      \end{column}
    \end{columns}
  \end{center}
  \vfill
  {\tiny * corresponding author}
\end{frame}

\end{document}
